\documentclass[12pt]{article}
\usepackage{amsmath}
\usepackage{geometry}
\usepackage{graphicx}
\usepackage{booktabs}
\usepackage{amsfonts}
\geometry{a4paper, margin=1in}
\usepackage[utf8]{inputenc}
\usepackage{CJKutf8}

\begin{document}

\begin{CJK}{UTF8}{gbsn}

\title{利用霍尔效应测量亥姆霍兹线圈磁场的实验报告}
\author{}
\date{2025年5月25日}
\maketitle

\section{实验目的}
利用霍尔效应测量亥姆霍兹线圈和螺线管轴线上的磁感应强度 \( B \),验证磁场叠加原理(即左右线圈单独通电时的磁场之和是否等于同时通电时的磁场),并分析实验误差的大小、产生原因及消减方法。

\section{实验原理}
实验基于霍尔效应,通过测量霍尔电压 \( u_H \) 计算磁感应强度 \( B \)。霍尔电压由公式计算:
\[
u_H = \frac{u_1 - u_2 + u_3 - u_4}{4}
\]
其中 \( u_1, u_2, u_3, u_4 \) 为霍尔探头在不同方向的测量电压(单位:mV)。磁感应强度 \( B \) 通过公式:
\[
B = \frac{u_H \times 10^{-3}}{K_H \cdot I_H} \times 1000 \, \text{(mT)}
\]
计算,其中 \( K_H = 845.6 \, \text{V} \cdot \text{A}^{-1} \cdot \text{T}^{-1} \),\( I_H = 2.50 \, \text{mA} = 2.50 \times 10^{-3} \, \text{A} \)。实验验证磁场叠加原理,即 \( B_{\text{左}} + B_{\text{右}} \approx B_{\text{(左+右)}} \),其中 \( B_{\text{左}} \) 和 \( B_{\text{右}} \) 为左、右线圈单独通电时的磁场,\( B_{\text{(左+右)}} \) 为理论值,\( B_{\text{左+B右}} \) 为实验测量值。

\section{实验数据}
实验测量了亥姆霍兹线圈和螺线管在轴线上不同位置 \( x \)(单位:cm)的磁感应强度,位置为:
\[
x = [0.0, 1.0, 2.0, 3.0, 4.0, 5.0, 6.0, 7.0, 8.0, 9.0, 10.0, 11.0, 12.0, 13.0, 14.0, 15.0]
\]
四组磁感应强度数据(单位:mT)如下:
\begin{itemize}
    \item \( B_{\text{左}} \)(左线圈单独通电): [0.128, 0.1645, 0.2162, 0.2942, 0.4025, 0.5289, 0.7159, 0.9325, 1.183, 1.3394, 1.3329, 1.1704, 0.9263, 0.6879, 0.4965, 0.36]
    \item \( B_{\text{右}} \)(右线圈单独通电): [0.4418, 0.6134, 0.8418, 1.0995, 1.2913, 1.3529, 1.2325, 0.9991, 0.7558, 0.5535, 0.4015, 0.3014, 0.2211, 0.1665, 0.1272, 0.1007]
    \item \( B_{\text{左+B右}} \)(左右线圈同时通电,实验值): [0.5698, 0.7779, 1.058, 1.3937, 1.6938, 1.8818, 1.9484, 1.9316, 1.9388, 1.8929, 1.7344, 1.4718, 1.1474, 0.8544, 0.6237, 0.4607]
    \item \( B_{\text{(左+右)}} \)(理论值,前16点): [0.5675, 0.7722, 1.0494, 1.3813, 1.6924, 1.8709, 1.9087, 1.9266, 1.9298, 1.9283, 1.9259, 1.926, 1.9174, 1.8867, 1.7352, 1.4613]
\end{itemize}

原始霍尔电压数据(单位:mV)如下:
\begin{itemize}
    \item \( u_1 \): [1.385, 1.740, 2.221, 2.767, 3.170, 3.300, 3.047, 2.558, 2.039, 1.613, 1.292, 1.082, 0.913, 0.796, 0.715, 0.659]
    \item \( u_2 \): [-0.970, -1.350, -1.837, -2.380, -2.785, -2.915, -2.658, -2.154, -1.645, -0.900, -0.686, -0.515, -0.407, -0.325, -0.260, -0.200]
    \item \( u_3 \): [0.902, 1.251, 1.733, 2.275, 2.700, 2.825, 2.575, 2.072, 1.562, 1.130, 0.819, 0.599, 0.426, 0.325, 0.230, 0.171]
    \item \( u_4 \): [-0.479, -0.846, -1.327, -1.875, -2.264, -2.400, -2.142, -1.665, -1.145, -0.725, -0.384, -0.182, -0.016, 0.120, 0.194, 0.239]
\end{itemize}

\section{误差分析}

\subsection{误差大小}
为验证磁场叠加原理,比较实验值 \( B_{\text{左+B右}} \)(左右线圈同时通电)和理论值 \( B_{\text{(左+右)}} \)(基于左、右线圈单独磁场之和)。差值和相对误差如下表所示:

\begin{table}[h]
\centering
\caption{\( B_{\text{左+B右}} \) 与 \( B_{\text{(左+右)}} \) 的差值与相对误差}
\begin{tabular}{ccc}
\toprule
\( x \) (cm) & 差值 (mT) & 相对误差 (\%) \\
\midrule
0.0 & 0.0023 & 0.41 \\
1.0 & 0.0057 & 0.74 \\
2.0 & 0.0086 & 0.82 \\
3.0 & 0.0124 & 0.90 \\
4.0 & 0.0014 & 0.08 \\
5.0 & 0.0109 & 0.58 \\
6.0 & 0.0397 & 2.08 \\
7.0 & 0.0050 & 0.26 \\
8.0 & 0.0090 & 0.47 \\
9.0 & -0.0354 & -1.84 \\
10.0 & -0.1915 & -9.94 \\
11.0 & -0.4542 & -23.58 \\
12.0 & -0.7700 & -40.00 \\
13.0 & -1.0323 & -54.71 \\
14.0 & -1.1115 & -64.06 \\
15.0 & -1.0006 & -68.49 \\
\bottomrule
\end{tabular}
\end{table}

误差分析:
\begin{itemize}
    \item \textbf{差值范围}:差值从 0.0014 mT(\( x = 4.0 \))到 -1.1115 mT(\( x = 14.0 \)),绝对值最大为 1.1115 mT。
    \item \textbf{相对误差}:在 \( x = 0.0 \) 到 \( x = 9.0 \) 范围内,相对误差较小(0.08\% 到 2.08\%),表明实验值与理论值接近,磁场叠加原理在该范围内基本得到验证。在 \( x = 10.0 \) 到 \( x = 15.0 \) 范围内,相对误差显著增大(9.94\% 到 68.49\%),表明实验值偏低,叠加原理验证效果较差。
    \item \textbf{平均绝对误差}:计算前16点的平均绝对误差为 \( \frac{\sum |\text{差值}|}{16} \approx 0.2913 \, \text{mT} \),显示整体误差较大,尤其在远场区域。
\end{itemize}

\subsection{误差产生原因}
误差可能来源于以下方面:
\begin{enumerate}
    \item \textbf{测量误差}:
        \begin{itemize}
            \item \textbf{霍尔电压测量}:\( u_1, u_2, u_3, u_4 \) 的测量受电压表分辨率(可能为0.01 mV或更低)或环境电磁噪声影响,导致 \( u_H \) 计算不精确,尤其在 \( x \) 较大时,电压值较小,相对误差放大。
            \item \textbf{位置误差}:探头在 \( x \) 轴上的定位可能存在偏差(如手动移动探头导致的0.1 cm误差),影响磁场测量精度。
        \end{itemize}
    \item \textbf{仪器校准}:
        \begin{itemize}
            \item 霍尔探头的灵敏度 \( K_H = 845.6 \, \text{V} \cdot \text{A}^{-1} \cdot \text{T}^{-1} \) 可能未精确校准,或随温度变化漂移。
            \item 电流 \( I_H = 2.50 \, \text{mA} \) 的稳定性可能受电源波动影响,导致 \( B \) 计算偏差。
        \end{itemize}
    \item \textbf{环境因素}:
        \begin{itemize}
            \item 外部磁场(如地磁场,约0.05 mT)或实验室设备干扰可能影响测量,尤其在 \( x = 10.0 \) 到 \( x = 15.0 \) 的远场区域,磁场较弱(0.46–1.47 mT),外部干扰占比更高。
            \item 温度变化可能影响霍尔探头灵敏度或线圈电阻,改变实际电流或磁场。
        \end{itemize}
    \item \textbf{理论模型偏差}:
        \begin{itemize}
            \item 理论值 \( B_{\text{(左+右)}} \) 假设亥姆霍兹线圈产生均匀磁场,但在 \( x = 10.0 \) 到 \( x = 15.0 \) 的远场区域,实际磁场可能显著衰减,导致理论值偏高。
            \item 螺线管或亥姆霍兹线圈的几何参数(如线圈间距、匝数)可能与理论模型不完全一致。
        \end{itemize}
    \item \textbf{数据处理误差}:
        \begin{itemize}
            \item \( u_H \) 四舍五入到四位有效数字可能引入微小误差,累积到 \( B \) 计算中。
            \item \( B_{\text{(左+右)}} \) 的后四点(对应 \( x = 16.0 \) 到 \( x = 19.0 \))未使用,可能表明实验设计中 \( x \) 范围定义不一致。
        \end{itemize}
\end{enumerate}

\subsection{误差消减方法}
为提高测量精度和验证磁场叠加原理的准确性,提出以下改进措施:
\begin{enumerate}
    \item \textbf{提高测量精度}:
        \begin{itemize}
            \item 使用高分辨率电压表(如微伏级)测量 \( u_1, u_2, u_3, u_4 \),减小霍尔电压的测量误差。
            \item 采用精密定位装置(如数控滑轨,精度达0.01 cm)确保探头在 \( x \) 轴上的准确定位。
            \item 增加每点测量次数(如5次),取平均值以减小随机误差。
        \end{itemize}
    \item \textbf{仪器校准}:
        \begin{itemize}
            \item 在实验前使用标准磁场源校准霍尔探头,验证 \( K_H \) 的准确性。
            \item 使用高稳定电流源(如恒流源,波动小于0.01 mA)确保 \( I_H \) 的稳定性。
        \end{itemize}
    \item \textbf{控制环境因素}:
        \begin{itemize}
            \item 在屏蔽室或法拉第笼中进行实验,屏蔽地磁场和其他电磁干扰。
            \item 控制实验环境温度(如恒温25°C),避免霍尔探头或线圈性能变化。
        \end{itemize}
    \item \textbf{改进理论模型}:
        \begin{itemize}
            \item 使用有限元分析软件模拟亥姆霍兹线圈的实际磁场分布,考虑线圈几何和远场衰减效应,修正 \( B_{\text{(左+右)}} \) 的理论值。
            \item 测量线圈的实际参数(如匝数、间距、半径),确保理论模型与实验设置一致。
        \end{itemize}
    \item \textbf{优化数据处理}:
        \begin{itemize}
            \item 保留更高精度的中间结果(如 \( u_H \) 保留6位小数),在最后计算 \( B \) 时再四舍五入。
            \item 统一 \( x \) 范围,明确是否需要测量 \( x > 15.0 \) 的数据,或补充对应 \( x = 16.0 \) 到 \( x = 19.0 \) 的实验值。
        \end{itemize}
\end{enumerate}

\section{验证磁场叠加原理}
磁场叠加原理要求 \( B_{\text{左}} + B_{\text{右}} = B_{\text{(左+右)}} \)。实验值 \( B_{\text{左+B右}} \) 与理论值 \( B_{\text{(左+右)}} \) 在 \( x = 0.0 \) 到 \( x = 9.0 \) 范围内较为接近(相对误差0.08\%–2.08\%),验证了磁场叠加原理的正确性。然而,在 \( x = 10.0 \) 到 \( x = 15.0 \) 范围内,实验值显著低于理论值(相对误差9.94\%–68.49\%),可能由于远场区域磁场衰减、外部干扰或测量误差导致原理验证失效。

\section{结论}
实验通过霍尔效应测量了亥姆霍兹线圈和螺线管轴线上的磁感应强度,验证了磁场叠加原理。在 \( x = 0.0 \) 到 \( x = 9.0 \) 范围内,实验值与理论值接近,相对误差较小(0.08\%–2.08\%),成功验证叠加原理。在 \( x = 10.0 \) 到 \( x = 15.0 \) 范围内,误差显著增大(9.94\%–68.49\%),表明远场测量存在较大偏差。误差可能来源于测量精度、仪器校准、环境干扰、理论模型偏差和数据处理。建议通过提高测量精度、校准仪器、屏蔽干扰、改进理论模型和优化数据处理来减小误差。未来可扩展 \( x \) 范围,增加测量点,进一步验证磁场分布规律。

\end{CJK}
\end{document}